\documentclass[ ../main.tex]{subfiles}
\providecommand{\mainx}{..}
\begin{document}
\section{Oblivious maps}
\label{sec:map}
A set is given by the following definition.
\begin{definition}
A set is an unordered collection of distinct elements from a universe of elements.
\end{definition}
We are interested in the \emph{countable sets}. A countable set is a \emph{finite set} or a \emph{countably infinite set}. A \emph{finite set} has a finite number of elements. For example,
\[
    \St = \{ 1, 3, 5 \}
\]
is a finite set with three elements. A \emph{countably infinite set} can be put in one-to-one correspondence with the set of natural numbers $\mathbb{N}$ given by
\begin{equation}
    \mathbb{N} = \{1,2,3,4,5,\ldots\}\,.
\end{equation}

The cardinality of a set $\St$ is a measure of the number of elements in the set, denoted by
\begin{equation}
    \card{\mathbb{\St}}\,.
\end{equation}
The cardinality of a \emph{finite set} is a non-negative integer and counts the number of elements in the set, e.g.,
\[
    \card{\left\{ 1, 3, 5\right\}} = 3\,.
\]

A map represents a \emph{many}-to-\emph{one}. A \emph{finite map} is given by the following definition.
\begin{definition}
A \emph{finite map} $\Mt \in \mathbb{X} \times \mathbb{Y}$ is a \emph{finite set} of ordered \emph{key}-\emph{value} pairs where \emph{keys} are from the universe of elements $\mathbb{X}$ and \emph{values} are from the universe of elements $\mathbb{Y}$. Additionally, a \emph{key} may appear at most once in the map.
\end{definition}

The key-value pairs of a \emph{countable map} can be put in one-to-one correspondence with the natural numbers. The \emph{cardinality} of a \emph{map} $\mathbb{T}$ is a measure of the number of \emph{key}-{value} pairs in the set, denoted by
\begin{equation}
    \card{\mathbb{T}}\,.
\end{equation}
The cardinality of a \emph{finite map} is a non-negative integer and counts the number of \emph{key}-\emph{value} pairs in the set, e.g.,
\[
    \card{\left\{ (1, a), (3, a), (5, b) \right\}} = 3\,.
\]

A \emph{value} is accessed by its associated \emph{key}. The value associated with key $k$ is denoted by
\begin{equation}
    \Mt[k]\,.
\end{equation}
The inverse operation of accessing the key from its associated value is not supported.

A function
\begin{equation}
    \Fun \colon \X \mapsto \Y
\end{equation}
defined over a \emph{finite} domain $\X$ may be represented by a \emph{table} where each element (input) in $\X$ is associated with some \emph{arbitrary} element (output) in $\Y$. For instance, \Cref{tbl:tabfunc} depicts a function over a finite domain of $n$ elements, where each input $x_j \in \X$ is associated with a single output $y_j \in \Y$, i.e., $y_j = \Fun(x_j)$.
\begin{table}
\centering
\caption{Function $\operatorname{f} \colon \X \mapsto \Y$\\defined over a \emph{finite} domain $\X$}
\label{tbl:tabfunc}
\begin{tabular}{|c c|} 
\hline
Input ($\mathbb{X}$) & Output ($\mathbb{Y}$)\\
\hline
    $x_1$ & $y_1$\\
    $x_2$ & $y_2$\\
    $x_3$ & $y_3$\\
    $\vdots$ & $\vdots$\\
    $x_n$ & $y_n$\\
\hline
\end{tabular}
\end{table}




\begin{table}
\centering
\caption[short title]{Function $\operatorname{f} \colon \X \mapsto \Y$\\defined over a \emph{finite} domain $\X$ such that if $y_j = \operatorname{f}(x_j)$, $b_j = \Encode_{\Y}(\operatorname{f}(x_j)) = b_j$ and $y_j = \Decode_{\Y}(b_j)$.}
\label{tbl:tabfunc}
\begin{tabular}{|c c c|} 
\hline
Input ($\mathbb{X}$) & Output ($\mathbb{Y}$) & Encoding ($\cisb$)\\
\hline
    $x_1$ & $y_1$ & $b_1$\\
    $x_2$ & $y_2$ & $b_2$\\
    $x_3$ & $y_3$ & $b_3$\\
    $\vdots$ & $\vdots$ & $\vdots$\\
    $x_n$ & $y_n$ & $b_n$\\
\hline
\end{tabular}
\end{table}



The oblivious map is given by the following definition.
\begin{definition}
The \gls{gls-obliviousmap} abstract data type is a specialization of the approximate map. We denote an oblivious map of $\Mt$ by $\Mpp$. An \emph{oblivious map} $\Mpp$ of $\Mt$ must satisfy the following conditions:
\begin{enumerate}
    \item $\AT{f} \colon \AT{X \mapsto Y}$ is a random approximate map of $f \colon X \mapsto Y$ with a specified false positive rate $\varepsilon$.
    \item There is no way to efficiently iterate over the domain of definition $X$.
    \item If an element of $x \in X$ is not in the domain of definition, $f(x)$ is a random oracle over $Y$
    \item A particular mapping $y = f(x)$ may only be learned by evaluating the function $f(x)$.
    \item Any estimator of the cardinality of the domain of definition that only uses the information in an object's binary representation may only be able determine an approximate specifiable upper-bound $u \geq m$ as a function of the entropy and the \emph{efficiency} of the representation, i.e., the most efficient representation has a very precise estimator of the cardinality of the domain of definition.
\end{enumerate}
\end{definition}
In an \emph{oblivious map}, a mapping (row in the table) is only learned upon request.

By \cref{dummyref}, the space complexity of the \emph{oblivious map} with a false positive rate $\varepsilon > 0$ and a false negative rate $\eta = 0$ is given by the following postulate.
\begin{postulate}
Space complexity
\begin{equation}
    -\log_2 \varepsilon + \mu \; \si{bits \per value}\,,
\end{equation}
where $\varepsilon$ is the false positive rate and $\mu$ is the mean \emph{marginal} entropy that a positive key maps to.
\end{postulate}

The cardinality of $\Mt$ is uniformly distributed between $0$ and The upper-bound $u$. Thus, the entropy of the cardinality is $\log_2(u+1)$. Thus, entropy may be traded for efficiency. For the optimal oblivious map, the relative efficiency to an approximate map where the entropy of the values in the key-value pairs is on average $\mu$ per value is given by
\begin{equation}
    \frac{\log_2 \varepsilon + \mu}{\frac{u}{m}\log_2 \varepsilon + \mu}\,.
\end{equation}
If $u = m$, then its relative efficiency is $1$. If $\mu = 0$, e.g., an approximate set, then the relative efficiency is given by
\begin{equation}
    \frac{m}{u}\,.
\end{equation}

\subsection{Abstract data type}
A \emph{type} is a set and the elements of the set are called the \emph{values} of the type. An \emph{abstract data type} is a type and a set of operations on values of the type. For example, the \emph{integer} abstract data type is the set of all integers and a set of standard operations such as addition and subtraction.

A \emph{data structure} is a particular way of organizing data and may implement one or more abstract data types. An \emph{immutable} data structure has static state; once constructed, its state does not change until it is destroyed. The abstract data type of the \emph{immutable} approximate map is given by the following definition..
\begin{definition}
The \emph{static map} is an abstract data type of the \emph{immutable} map. Any of the following operations may be supported:
\begin{enumerate}
    \item \HasKey{$\Mp$,$k$}. Returns $\True$ if $(k,v') \in \Mt$ for any $v'$. Otherwise, \False.
    \item \Find{$\Mp$,$k$}. Returns $v$ if $(k,v) \in \Mt$ for any $v$. Otherwise, returns some unspecified $v'$ with probability $\varepsilon$ and $\nullvalue$ with probability $1 - \varepsilon$.
    \item \Cardinality{$\Mp$}. Returns the \emph{expected} cardinality $(1 - \varepsilon) \card{\Mt} + \varepsilon \card{\mathbb{X}}$.
    \item \Count{$\Mt$}. Returns $\card{\Mt}$.
    \item \fprate{$\Mp$}. Returns the \emph{\gls{gls-fprate}} of $\Mp$.
\end{enumerate}
\end{definition}
\end{document}