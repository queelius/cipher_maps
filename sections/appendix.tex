\documentclass[ ../main.tex]{subfiles}
\providecommand{\mainx}{..}
\begin{document}
\section{This section needs title}
For a particular $n$, each $b \in \cisb_n$ may fail and therefore $n$ is uncertain and takes on particular values with probabilities given by the following theorem.
\begin{theorem}
\label{thm:N_pmf}
The \gls{gls-shs} generator given by \Cref{alg:ph} finds a bit string of random length $\rv{N} = n$ that results in a perfect collision on $\St$ with a probability mass function given by
\begin{equation}
\label{eq:N_pmf}
    \pmf_{\rv{N}}(n \given m, \varepsilon, \mu) = q^{2^n-1}\left(1 - q^{2^n}\right)\,,
\end{equation}
where $q = 1 - p(m,\varepsilon,\mu)$, $m = \card{\St}$, and $\varepsilon$ is the false positive rate.
\end{theorem}
\begin{proof}
Each iteration of the loop in \Cref{alg:shmap} has a collision test which is Bernoulli distributed with a probability of success $p(m,\varepsilon,\mu)$. We are interested in the random length $\rv{N}$ of the bit string when this outcome occurs.

For the random variable $\rv{N}$ to realize a value $n$, every bit string smaller than length $n$ must fail and a bit string of length $n$ must succeed. There are $2^n-1$ bit strings smaller than length $n$ and each one fails with probability $q$, and so by the product rule the probability that they all fail is given by
\begin{equation}
\label{eq:N_pmf_1}
    q^{2^n - 1}\,.
\end{equation}

Given that every bit string smaller than length $n$ fails, what is the probability that every bit string of length $n$ fails? There are $2^n$ bit strings of length $n$, each of which fails with probability $q$ as before and thus by the product rule the probability that they all fail is $q^{2^n}$, whose complement, the probability that not all bit strings of length $n$ fail, is given by
\begin{equation}
\label{eq:N_pmf_2}
    1 - q^{2^n}\,.
\end{equation}
By the product rule, the probability that every bit string smaller than length $n$ fails and a bit string of length $n$ succeeds is given by the product of (\ref{eq:N_pmf_1}) and (\ref{eq:N_pmf_2}),
\begin{equation}
\label{eq:prob_is_mass}
    q^{2^n-1}\left(1 - q^{2^n}\right)\,.
\end{equation}
For \Cref{eq:prob_is_mass} to be a probability mass function, two conditions must be met. First, its range must be a subset of $[0,1]$. Second, the summation over its domain must be $1$.

The first case is trivially shown by the observation that $q$ is a positive number between $0$ and $1$ and therefore any non-negative power of $q$ is positive number between $0$ and $1$.

The second case is shown by calculating the infinite series
\begin{align}
    S &= \sum_{n=0}^{\infty} q^{2^n-1}\left(1-q^{2^n}\right)\\
      &= \sum_{n=0}^{\infty} q^{2^n-1} - q^{2^{n+1}-1}\,.
\end{align}
Explicitly evaluating this series for the first $4$ terms reveals a telescoping sum given by
\begin{equation}
    S = (1 - q) + (q - q^3) + (q^3 - q^7) + (q^7 - q^{15}) + \cdots\,,
\end{equation}
where everything cancels except $1$.
\end{proof}

The expected encoding size is given by the following theorem.
\begin{theorem}
\label{thm:open_expect}
The expected \emph{in-place} (no decoding step) encoding size of the \gls{gls-shs} $\Sp$ is given by
\begin{equation}
    \mathcal{O}(1) + \sum_{n=1}^{\infty} q^{2^n-1}\,,
\end{equation}
where $q=1 - p(m,\varepsilon,\mu)$, $m = \card{\St}$, $\varepsilon$ is the false positive rate, and $\mu$ is the average bit length of the values in the key-value pairs.
\end{theorem}
\begin{proof}
The hash function $\hash_{\St}$ given by \Cref{alg:ph} finds a bit string of random length $\rv{N} = n$. Thus the encoding size is a discrete random variable given by
\begin{equation}
    \mathcal{O}(1) + \rv{N}\,,
\end{equation}
where the constant is the space required to encode the random oracle. The expected encoding size is the expectation given by
\begin{align}
    \mathcal{O}(1) + \expectation[\rv{N}]\,.
\end{align}
By \Cref{thm:N_pmf}, the random variable $\rv{N}$ has a probability mass function given by
\begin{equation*}
    \rv{N} \sim \pmf_{\rv{N}}(n \given m,r) = q^{2^n-1}\left(1 - q^{2^n}\right)\,,
    \tag{\ref{eq:N_pmf} revisited}
\end{equation*}
and thus the expectation of $\rv{N}$ is given by
\begin{align}
    \expectation\!\left[\rv{N}\right]
    &= \sum_{n=0}^{\infty} n q^{2^n-1}\left(1 - q^{2^n}\right)\\
    &= \sum_{n=0}^{\infty} n \left(q^{2^n-1} - q^{2^{n+1}-1}\right)\,.
\end{align}
Explicitly evaluating this series for the first $4$ terms reveals a converging sum given by
\begin{align}
    \expectation\!\left[\rv{N}\right]
        &= 0 + (q - q^3) + 2(q^3 - q^7) + 3(q^7 - q^{15}) + \cdots\\
        &= q + q^3 + q^7 + q^{15} + \cdots\,.
\end{align}
\end{proof}
\end{document}