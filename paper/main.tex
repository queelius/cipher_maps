\documentclass{article}
\usepackage[english]{babel}
\usepackage{graphicx}
\graphicspath{img}
\usepackage[activate={true,nocompatibility},final,tracking=true,kerning=true,spacing=true,factor=1100,stretch=10,shrink=10]{microtype}
\microtypecontext{spacing=nonfrench}
\usepackage[]{algorithm2e}
\usepackage{caption}
\usepackage{bbm}
\usepackage{mathtools}
\usepackage{commath}
\usepackage{amsmath}
\usepackage{subfiles}
\usepackage[xindy,toc,acronyms,symbols]{glossaries}
\usepackage{latexsym}
\usepackage{amsthm}
\usepackage{amssymb}
\usepackage{pgfplots}
\usepackage{wasysym}
\usepackage{hyperref}
\usepackage{tikz}
\usepackage{tikzscale}
\usepackage[square,numbers]{natbib}
\usepackage[utf8]{inputenc}
\usepackage[T1]{fontenc}
\usepackage{cleveref}
\usepackage{siunitx}
\usepackage[section]{placeins}

\hypersetup{
    pdftitle={The oblivious map abstract data type},             % title
    pdfauthor={Alexander Towell},               % author
    pdfsubject={computer science},              % subject of the document
    pdfkeywords={
        probabilistic data structure,
        approximate set,
        approximate domain},                    % keywords
    colorlinks=false,                           % false: boxed links;
    citecolor=green,                            % color of links to 
    filecolor=magenta,                          % color of file links
    urlcolor=green                              % color of external
}



\title
{
    The \emph{oblivious map} abstract data type
}
\author
{
    Alexander Towell\\
    \texttt{atowell@siue.edu}
}
\date{}

\begin{document}
\maketitle
\begin{abstract}
We define the semantics of the \emph{oblivious map}, which is a \emph{probabilistic} map that \emph{approximates} another map with a false positive rate $\varepsilon$ and false negative rate $\eta$ with the \emph{positive} ($\eta = 0$) and negative ($\varepsilon = 0$) oblivious maps being special complementary cases. As approximate maps, all the results that apply to approximate maps apply to them. We derive a theoretical data structure that implements the abstract data type of the \emph{oblivious map}, denoted the \gls{gls-shmap}. We explore its theoretical properties and conjecture that it obtains the lower-bound on space complexity when the inputs and outputs are of arbitrary length with \emph{false positive} and \emph{false negative} domain error rates given respectively by $\varepsilon$ and $\eta$. Additionally, we demonstrate an application, rank-ordered \emph{Encrypted Search}.
\end{abstract}

\tableofcontents
\listoffigures
\listofalgorithms

\section{Introduction}
\label{sec:intro}

Intro here.


\section{Oblivious Maps}
\label{sec:map}
A set is an unordered collection of distinct elements, typically from some implicitly understood universe.
A countable set is a \emph{finite set} or a \emph{countably infinite set}. A \emph{finite set} has a finite number of elements, such as $\{ 1, 3, 5 \}$, and 
a \emph{countably infinite set} can be put in one-to-one correspondence with the set of natural numbers, $\{1,2,3,4,5,\ldots\}$.
The cardinality of a set $\mathbb{A}$, denoted by $|\mathbb{A}|$, is a measure on the number of elements in the set, e.g., $|\{a,b\}|=2$.

A map represents a \emph{many}-to-\emph{one} relationship. A map that associates elements in $\mathbb{X}$ to elements in $\mathbb{Y}$ has a type denoted by $\mathbb{X}\mapsto\mathbb{Y}$.
For a map of type $\mathbb{X} \mapsto \mathbb{Y}$, we denote $\mathbb{X}$ the \emph{input} and $\mathbb{Y}$ the output.
Typically, it is relatively easy to find which output is associated with a given input, but the inverse operation, determining which inputs are associated with a given output is computationally harder.
Of course, this is not necessarily the case, and mathematically the map is just a one-to-many relation over $\mathbb{X} \times \mathbb{Y}$.

For instance, \Cref{tbl:tabfunc} depicts a function over a finite domain of $n$ elements, where each input $x \in X$ is associated with a single output $y \in Y$, i.e., $y = \Fun(x)$.

\begin{table}
    \centering
    \caption{Function $\operatorname{negate} : \{0,1\} \mapsto \{0,1\}$}
    \label{tbl:tabfunc}
    \begin{tabular}{|c c|} 
        \hline
        Input ($\{0,1\}$) & Output ($\{0,1\}$)\\
        \hline
        $0$ & $1$\\
        $1$ & $0$\\    
        \hline
    \end{tabular}
\end{table}

The input does not need to be a simple set like natural numbers, but rather can be any type of set, such as a set of pairs as given in \Cref{tbl:tabfun2}.

\begin{table}
\centering
\caption{Function $\operatorname{f} : \X \mapsto \Y$\\defined over a \emph{finite} domain $\X$}
\label{tbl:tabfunc2}
\begin{tabular}{|c c|} 
\hline
Input ($\{0,1\} \times \{0,1\}$) & Output ($\{0,1\}$)\\
\hline
    $(0,0)$ & $0$\\
    $(0,1)$ & $1$\\
    $(1,0)$ & $1$\\
    $(1,1)$ & $0$\\    
\hline
\end{tabular}
\end{table}


Since we are interested in constructing maps in computer memory, we must have some way to represent them.
One technique may be given by the following table.

\begin{table}
\centering
\caption{Function $\operatorname{f} : \{0,1\} \mapsto \{0,1\}$ where $b : \{0,1\} \mapsto \{0,1\}^*$ is the encoder for $\mathbb{Y}$}
\label{tbl:tabfunc}
\begin{tabular}{|c c c|} 
\hline
Input ($\mathbb{X}$) & Output ($\mathbb{Y}$) & Encoding ($\cisb$)\\
\hline
    $x_1$ & $y_1$ & $b_1$\\
    $x_2$ & $y_2$ & $b_2$\\
    $x_3$ & $y_3$ & $b_3$\\
    $\vdots$ & $\vdots$ & $\vdots$\\
    $x_n$ & $y_n$ & $b_n$\\
\hline
\end{tabular}
\end{table}



The oblivious map is given by the following definition.
\begin{definition}
The \gls{gls-obliviousmap} abstract data type is a specialization of the approximate map. We denote an oblivious map of $\Mt$ by $\Mpp$. An \emph{oblivious map} $\Mpp$ of $\Mt$ must satisfy the following conditions:
\begin{enumerate}
    \item $\AT{f} : \AT{X \mapsto Y}$ is a random approximate map of $f : X \mapsto Y$ with a specified false positive rate $\varepsilon$.
    \item There is no way to efficiently iterate over the domain of definition $X$.
    \item If an element of $x \in X$ is not in the domain of definition, $f(x)$ is a random oracle over $Y$
    \item A particular mapping $y = f(x)$ may only be learned by evaluating the function $f(x)$.
    \item Any estimator of the cardinality of the domain of definition that only uses the information in an object's binary representation may only be able determine an approximate specifiable upper-bound $u \geq m$ as a function of the entropy and the \emph{efficiency} of the representation, i.e., the most efficient representation has a very precise estimator of the cardinality of the domain of definition.
\end{enumerate}
\end{definition}
In an \emph{oblivious map}, a mapping (row in the table) is only learned upon request.

By \cref{dummyref}, the space complexity of the \emph{oblivious map} with a false positive rate $\varepsilon > 0$ and a false negative rate $\eta = 0$ is given by the following postulate.
\begin{postulate}
Space complexity
\begin{equation}
    -\log_2 \varepsilon + \mu \; \si{bits \per value},
\end{equation}
where $\varepsilon$ is the false positive rate and $\mu$ is the mean \emph{marginal} entropy that a positive key maps to.
\end{postulate}

The cardinality of $\Mt$ is uniformly distributed between $0$ and The upper-bound $u$. Thus, the entropy of the cardinality is $\log_2(u+1)$. Thus, entropy may be traded for efficiency. For the optimal oblivious map, the relative efficiency to an approximate map where the entropy of the values in the key-value pairs is on average $\mu$ per value is given by
\begin{equation}
    \frac{\log_2 \varepsilon + \mu}{\frac{u}{m}\log_2 \varepsilon + \mu}\,.
\end{equation}
If $u = m$, then its relative efficiency is $1$. If $\mu = 0$, e.g., an approximate set, then the relative efficiency is given by
\begin{equation}
    \frac{m}{u}\,.
\end{equation}

\subsection{Abstract data type}
A \emph{type} is a set and the elements of the set are called the \emph{values} of the type. An \emph{abstract data type} is a type and a set of operations on values of the type. For example, the \emph{integer} abstract data type is the set of all integers and a set of standard operations such as addition and subtraction.

A \emph{data structure} is a particular way of organizing data and may implement one or more abstract data types. An \emph{immutable} data structure has static state; once constructed, its state does not change until it is destroyed. The abstract data type of the \emph{immutable} approximate map is given by the following definition..
\begin{definition}
The \emph{static map} is an abstract data type of the \emph{immutable} map. Any of the following operations may be supported:
\begin{enumerate}
    \item \HasKey{$\Mp$,$k$}. Returns $\True$ if $(k,v') \in \Mt$ for any $v'$. Otherwise, \False.
    \item \Find{$\Mp$,$k$}. Returns $v$ if $(k,v) \in \Mt$ for any $v$. Otherwise, returns some unspecified $v'$ with probability $\varepsilon$ and $\nullvalue$ with probability $1 - \varepsilon$.
    \item \Cardinality{$\Mp$}. Returns the \emph{expected} cardinality $(1 - \varepsilon) \card{\Mt} + \varepsilon \card{\mathbb{X}}$.
    \item \Count{$\Mt$}. Returns $\card{\Mt}$.
    \item \fprate{$\Mp$}. Returns the \emph{\gls{gls-fprate}} of $\Mp$.
\end{enumerate}
\end{definition}





\section{The \emph{Singular Hash Map}}
The \emph{Singular Hash Set} is a theoretical data structure that provides an \emph{optimal} implementation of the \emph{oblivious map} abstract data type. A more practical implementation is given by the \emph{Perfect Map Filter}\cite{pmf}.

We consider sets in which the universe of elements is given by the \emph{countably infinite} set of all bit strings.
\begin{definition}
The countably infinite set of all bit strings is denoted by
\begin{equation}
    \cisb = \left\{b : b \in \left\{0,1\right\}^*\right\},
\end{equation}
where $*$ denotes all non-negative integers.
\end{definition}
A \emph{finite} subset of $\cisb$ is given by the following definition.
\begin{definition}
The \emph{finite set} of all bit strings of length $n$ is denoted by
\begin{equation}
    \cisb_n = \left\{b : b \in \left\{0,1\right\}^n\right\}
\end{equation}
with a cardinality given by
\begin{equation}
    \card{\cisb_n} = 2^n\,.
\end{equation}
\end{definition}

A \emph{hash function} is related to countable sets $\cisb$ and $\cisb_n$ and is given by the following definition.
\begin{definition}
A \emph{hash function} $\hash : \cisb \mapsto \cisb_n$ is a function such that all bit strings of arbitrary-length are mapped (hashed) to bit strings of fixed-length $n$.
\end{definition}
For a given $x \in \cisb$, $y = \hash{x} \in \cisb_n$ is denoted the \emph{hash} of $x$.


The \gls{gls-shmap} depends on a random oracle as given by the following definition.
\begin{definition}
\label{def:randomoracle}
A random oracle, denoted by $\ro : \cisb \mapsto \cisb_n$, is a theoretical \emph{hash function} whose output is uniformly distributed over the elements of $\cisb_n$.
\end{definition}
\begin{assumption}
The hash function $\hash$ approximates a random oracle $\ro$ and has a space complexity given by $\mathcal{O}(1)$.
\end{assumption}

The bit length of objects is given by the following definition.
\begin{definition}
The bit length of an object $x$ is denoted by
\begin{equation}
    \BL(x)\,.
\end{equation}
\end{definition}
For example, the bit length of an object $x \in \cisb_n$ is given by $\BL(x) = n$.

\begin{algorithm}[h]
    \caption{Theoretical implementation of \protect\MakeApproxMap}
    \label{alg:shmap}
    \SetKwProg{func}{function}{}{}
    \KwIn
    {
        $\Fun = \{\left(x,\Fun(x)\right) : x \in \X \}$ is the function defined as a set of ordered pairs, $\varepsilon$ is the \emph{false positive} domain error rate, and $\eta$ is the \emph{false negative} domain error rate.
    }
    \KwOut
    {
        An \emph{oblivious} function $\Afun$ with an approximate domain with a \emph{false positive} rate $\varepsilon$ and \emph{false negative} rate less than or equal to $\eta$.
    }
    \SetKwData{found}{found}
    \SetKwData{truecode}{true\_code}
    \SetKwData{code}{test\_code}
    \func{\MakeApproxMap{$\Fun$, $\varepsilon$}}
    {
        $\Mt_{\cisb \times \cisb} \gets \left\{\left(\Encode_{\X \mapsto \cisb}\!\left(x\right),\Encode_{\Y \mapsto \cisb}(\Fun(x))\right) : x \in \X\right\}$\;
        $p \gets (1 - \eta) \card{\X}$\;
        $k \gets -\log_2 \varepsilon$\;
        \For{$n \gets 0$ \KwTo $\infty$}
        {
            $\found \gets \True$\;
            \For{$j \gets 1$ \KwTo $2^n$}
            {
                $b_n \gets$ draw random bit string of length $n$ from $\cisb_n$ without replacement\;
                $\hash(z) \equiv \ro(z \catop b_n)$\;
                \tcp{A minimum of $p$ inputs must have the same
                hash.}
                $\mathbb{M} \gets \text{all subsets of $p$ elements from $\Mt_{\cisb \times \cisb}$}$\;
                \For{$\mathbb{P} \in \mathbb{M}$}
                {
                    \tcp{Let the $p$ elements in $\mathbb{P}$ be denoted by $\left(x^{(1)}, y^{(1)}\right),\ldots,\left(x^{(p)},y^{(p)}\right)$.}
                    $h_k \gets \hash(x) \mod k$\;
                    \For{$i \gets 2$ \KwTo $p$}
                    {
                        \If{$\hash(x^{(i)}) \mod k \neq h_k$}
                        {
                            $\found \gets \False$\;
                        }
                        \tcp{$\Encode_{\mathbb{Y} \mapsto \cisb}$ is a {self-delimiting} coder so we only check that the first $\BL(y^{(i)})$ bits of the hash are equal to $y^{(i)}$.}
                        $\code \gets \hash(1 \catop x_i) \mod \BL(y^{(i)})$\;
                        \If{$\code \neq y^{(i)}$}
                        {
                            \label{line:selfterm}
                            $\found \gets \False$\;
                        }
                    }
                    \If{\found}
                    {
                        \tcp{This tuple codes the \gls*{gls-shmap}.}
                        \Return $(h_k,b_n)$\;
                    }
                }
            }
        }
    }
\end{algorithm}

\begin{algorithm}[h]
    \caption{Theoretical implementation of \protect\Find}
    \label{alg:shmapfind}
    \SetKwProg{func}{function}{}{}
    \KwIn
    {
        $\Mp \in \mathbb{X} \times \mathbb{Y}$ is the \emph{\gls{gls-shmap}} with a \emph{\gls{gls-fprate}} $\varepsilon$.
    }
    \KwOut
    {
        If $x \in \mathbb{X}$ is a key in $\Mt$, then the \emph{value} in $\mathbb{Y}$ mapped to by $x$ is returned. Otherwise, with probability $\varepsilon$, $k$ is a false positive and maps to a \emph{random} bit string and with probability $1-\varepsilon$, the \emph{null value} is returned.
    }
    
    \SetKwData{Code}{code}
    \SetKwData{Success}{success}
    \func{\Find{$\Mp$, $x$}}
    {
        \tcp{The \emph{\gls{gls-shmap}} is coded by the tuple $(h_1,b_n)$ as given by \Cref{alg:shmap}.}
        $\ell \gets \BL(h_1)$\;
        $\Code_x \gets \Encode_{\mathbb{X} \mapsto \cisb}(x)$\;
        \If{$\hash(\Code_x) \mod \ell \neq h_1$}
        {
            \Return \nullvalue\;
        }
        
        
        $\Code_{x'} \gets \hash(1 \catop x) \mod j$\;
        \tcp{We assume the true code associated with $x$ is the result of a \emph{self-delimiting} coder which can be decoded by $\Decode : \cisb \mapsto \mathbb{Y}$.}
        $(\Success, y) \gets \Decode_{\cisb \mapsto \mathbb{Y}}(\Code_y)$\;
        \If{\Success}
        {
            \tcp{Since $k$ may be a false positive, $v$ may be a random bit string with a length $\ell \in [c,N], 0 \leq c \leq N$, where $c$ depends upon the self-delimiting coder.}
            \Return $y$\;
        }
    }
\end{algorithm}


\begin{figure}
    \centering
    %%% Creator: Inkscape inkscape 0.92.1, www.inkscape.org
%% PDF/EPS/PS + LaTeX output extension by Johan Engelen, 2010
%% Accompanies image file 'shs.pdf' (pdf, eps, ps)
%%
%% To include the image in your LaTeX document, write
%%   \input{<filename>.tex}
%%  instead of
%%   \includegraphics{<filename>.pdf}
%% To scale the image, write
%%   \def\svgwidth{<desired width>}
%%   \input{<filename>.pdf_tex}
%%  instead of
%%   \includegraphics[width=<desired width>]{<filename>.pdf}
%%
%% Images with a different path to the parent latex file can
%% be accessed with the `import' package (which may need to be
%% installed) using
%%   \usepackage{import}
%% in the preamble, and then including the image with
%%   \import{<path to file>}{<filename>.pdf_tex}
%% Alternatively, one can specify
%%   \graphicspath{{<path to file>/}}
%% 
%% For more information, please see info/svg-inkscape on CTAN:
%%   http://tug.ctan.org/tex-archive/info/svg-inkscape
%%
\begingroup%
  \makeatletter%
  \providecommand\color[2][]{%
    \errmessage{(Inkscape) Color is used for the text in Inkscape, but the package 'color.sty' is not loaded}%
    \renewcommand\color[2][]{}%
  }%
  \providecommand\transparent[1]{%
    \errmessage{(Inkscape) Transparency is used (non-zero) for the text in Inkscape, but the package 'transparent.sty' is not loaded}%
    \renewcommand\transparent[1]{}%
  }%
  \providecommand\hashtatebox[2]{#2}%
  \ifx\svgwidth\undefined%
    \setlength{\unitlength}{214.63718569bp}%
    \ifx\svgscale\undefined%
      \relax%
    \else%
      \setlength{\unitlength}{\unitlength * \real{\svgscale}}%
    \fi%
  \else%
    \setlength{\unitlength}{\svgwidth}%
  \fi%
  \global\let\svgwidth\undefined%
  \global\let\svgscale\undefined%
  \makeatother%
  \begin{picture}(1,0.55063496)%
    \put(0.38,0.078){\color[rgb]{0,0,0}\makebox(0,0)[lb]{\smash{$h(x_2 + b) \mod 4$}}}%
    \put(0.08843285,0.51926543){\color[rgb]{0,0,0}\makebox(0,0)[lb]{\smash{set $\mathbb{S}$}}}%
    \put(0,0){\includegraphics[width=\unitlength,page=1]{fig_shs.pdf}}%
    \put(0.09242613,0.24054801){\color[rgb]{0,0,0}\makebox(0,0)[lb]{\smash{$x_1$}}}%
    \put(0.12846631,0.36240699){\color[rgb]{0,0,0}\makebox(0,0)[lb]{\smash{$x_3$}}}%
    \put(0.13833525,0.11371431){\color[rgb]{0,0,0}\makebox(0,0)[lb]{\smash{$x_2$}}}%
    \put(0.92073981,0.50635972){\color[rgb]{0,0,0}\makebox(0,0)[lt]{\begin{minipage}{0.53112884\unitlength}\raggedright \end{minipage}}}%
    \put(0,0){\includegraphics[width=\unitlength,page=2]{fig_shs.pdf}}%
    \put(0.41093171,0.41327568){\color[rgb]{0,0,0}\makebox(0,0)[lb]{\smash{$h(x_3 + b)$}}}%
    \put(0.35,0.275){\color[rgb]{0,0,0}\makebox(0,0)[lb]{\smash{$h(x_1 + b)}$}}}%
    \put(0,0){\includegraphics[width=\unitlength,page=3]{fig_shs.pdf}}%
    \put(0.81940596,0.29320931){\color[rgb]{0,0,0}\makebox(0,0)[lb]{\smash{$0\,1\,0\,0$}}}%
    \put(0.76349761,0.2128412){\color[rgb]{0,0,0}\makebox(0,0)[lb]{\smash{$\fprate = 2^{-4}$}}}%
  \end{picture}%
\endgroup%
    \caption{\emph{Singular Hash Map}}
    \label{fig:shmap}
\end{figure}

We consider a family of \emph{approximate maps} which are given by the output of the random oracle applied to the input $x \in \Mt$ concatenated with a bit string $b \in \cisb$ such that all $x \in \St$ map to the same hash. We describe the generative algorithm in \Cref{alg:ph}. Note that in \Cref{alg:shmap}, the concatenation of two bit strings $x$ and $y$ is denoted by $x \catop y$.


In \Cref{alg:shmap} on \Cref{line:selfterm}, the value $v_i$ has a \emph{self-terminating} encoding. Thus, we need only check that the bit string that encodes $v_i$ is equal to a hash of the key (concatenated with another $1$ and another bit string $b$).


The space required for the \gls{gls-shmap} found by \Cref{alg:shmap} is of the order of the length $n$ of the bit string $b$. Therefore, for space efficiency, the algorithm exhaustively searches for a bit string in the order of increasing size $n$.

\subsection{Theoretical analysis}
The \gls{gls-set} proves that the keys in $\Mp$ are an \emph{approximate set} of the keys in $\Mt$.

\begin{theorem}
The probability that a particular bit string $b \in \cisb$ successfully codes the \gls{gls-shmap} is given by
\begin{equation}
    p(m,\varepsilon,\mu) = \frac{\varepsilon^{m-1}}{2^{m \mu}}\,.
\end{equation}
\end{theorem}
\begin{proof}
In \Cref{alg:shmap}, for a particular bit string $b_n$, the joint probability that every key in $\Mt$ collides and for each key $k$ in $\Mt$, the concatenation of $1$ and $k$ has a hash in which the first $\BL(v)$ bits collide with mapped value $v$ is given by the following reasoning.

Suppose we have a set $\Mt = \{(k_1,v_1),\ldots,(k_m,v_m)\}$ and $k_1$ hashes to $y = \hash(k_1) \mod t$, where $\hash : \cisb^* \mapsto \cisb_N$ is a random hash function that uniformly distributes over its domain of $2^N$ possibilities and $N < \max{\BL(v_1),\ldots,\BL(v_m)}$.

The probability that $\hash(1 \catop k_1) \mod \BL(v_1) = v_1$ is given by
\begin{equation}
    \frac{1}{2^{\BL(v_1)}}\,.
\end{equation}
Since $y$ is a particular element in $\cisb_k$, the probability that $k_j, j \neq 1$, hashes to $y$ is given by
\begin{equation}
    \frac{1}{2^k}\,.
\end{equation}
The probability that $1 \catop k_j, j \neq 1$ hashes to $v_j$ is given by
\begin{equation}
    \frac{1}{2^{\BL(v_j)}}\,.
\end{equation}
Since $\hash$ is a random hash function, every one of these probabilities are independent, and thus the joint probability is just the product given by
\begin{align}
    p &= \frac{1}{2^{k(m-1)}} \prod_{j=1}^{m} \frac{1}{2^{\BL(v_j)}}\\
      &= \varepsilon^{m-1} \frac{1}{2^{\sum_{j=1}^{m} \BL(v_j)}}\,.
\end{align}
The average bit length per key is given by
\begin{equation}
    \mu = \frac{1}{m} \sum_{j=1}^{m} \BL(v_j)
\end{equation}
and thus the probability $p$ may be parameterized as
\begin{equation}
    p = \frac{\varepsilon^{m-1}}{2^{m \mu}}\,.
\end{equation}
\end{proof}

\begin{theorem}
Given \emph{false positive} and \emph{false negative} domain error rates given by $\varepsilon$ and $\eta$ respectively, and a mean bit length $\mu$ of encodings of elements in the image of $\Afun$, the \emph{expected} bit length of the Singular Hash Map asymptotically obtains a space complexity with a lower-bound given by
\begin{equation}
    -(1 - \eta) \log_2 \varepsilon + (1 - \eta) \mu \; \si{bits \per element},
\end{equation}
which occurs when the encodings have uniformly distributed bit lengths, and an upper-bound given by
\begin{equation}
    -(1 - \eta) \log_2 \varepsilon + \mu \; \si{bits \per element},
\end{equation}
which occurs when the encodings have degenerate bit lengths. The greater the entropy of the bit lengths, the smaller the expected bit length given a mean of $\mu$. 
\end{theorem}
\begin{proof}
The space required for the \gls{gls-shs} found by \Cref{alg:makeset} is of the order of the length $n$ of the bit string $b_n$. Therefore, for space efficiency, the algorithm exhaustively searches for a bit string in the order of increasing size $n$.

We are interested in the first case when a success occurs, which is a geometric distribution with probability of success $p$ as given by
\begin{equation}
    \rv{Q} \sim \geodist\!\left(p = \frac{\varepsilon^{m-1}}{2^{m \mu}}\right)\,.
\end{equation}
The expected number of trials for the geometric distribution is given by
\begin{equation}
\label{eq:exp_trials}
    \expectation\left[\rv{Q}\right] = \frac{2^{m \mu}}{\varepsilon^{m-1}}\,.
\end{equation}
By \Cref{def:mapping}, the \nth trial uniquely maps to a bit string of length $m = \lfloor \log_2 n \rfloor$. Thus, the expected bit length is given approximately by
\begin{equation}
    \expectation[\log_2 \rv{Q}] = \log_2 \frac{2^{m \mu}}{\varepsilon^{m-1}}\; \si{bits}\,.
\end{equation}
Dividing by $m$ and simplifying results in
\begin{equation}
    -\frac{m-1}{m} \log_2 \varepsilon + \mu\; \si{bits \per element}\,.
\end{equation}
Asymptotically, as $m \to \infty$, the bits per element goes to
\begin{equation}
    -\log_2 \varepsilon + \mu\,.
\end{equation}
\end{proof}

The smaller the bit lengths generated by the encoder, the smaller the \emph{Singular Hash Map}. However, optimal encoders \emph{reveal} information about the values in the \emph{Singular Hash Map}, which goes against the principle of the \emph{obvlivious function}. For instance, if each input $x$ mapped to a list of numbers, then an optimal encoder will generate codes that minimize the length of the list, say a Huffman code. However, by analyzing the Huffman codes produced by the encoder, we may infer the distribution of the lists. Thus, for the sake of maintaining an \emph{oblivious} state, the encoder should generate codes as though the values were \emph{uniformly} distributed.

\subsection{Entropy}
The entropy of the \gls{gls-shmap} is given by the following definition.



\section{Application: secure indexes in \emph{Encrypted Search}}
enables untrusted systems to map \glspl{gls-key} to \glspl{gls-value}
obliviously search documents on behalf of authorized users, where the search query, the confidential documents being searched over, and the search results are equivalent to a random bit sequence.

These requirements may not be fully met in most cases. In this paper, we consider the confidentiality of a document which can be obviously searched.

the untrusted system does not know what the user is searching for, nor what the confidential documents contain. an obvious search is one in which the untrusted system does

One of the earlier papers presented on \gls{es} \cite{Son00} observed that individuals may wish to outsource storage logistics to cloud storage providers but do not trust the providers with confidentiality requirements. 
To enable an untrusted system to search a confidential document on an authorized user's behalf, the untrusted system must have some sort searchable representation of the document. 

Rank-ordered search. Oblivious sets are more appropriate for Boolean search.



\section{This section needs title}
For a particular $n$, each $b \in \cisb_n$ may fail and therefore $n$ is uncertain and takes on particular values with probabilities given by the following theorem.
\begin{theorem}
\label{thm:N_pmf}
The \gls{gls-shs} generator given by \Cref{alg:ph} finds a bit string of random length $\rv{N} = n$ that results in a perfect collision on $\St$ with a probability mass function given by
\begin{equation}
\label{eq:N_pmf}
    \pmf_{\rv{N}}(n \given m, \varepsilon, \mu) = q^{2^n-1}\left(1 - q^{2^n}\right),
\end{equation}
where $q = 1 - p(m,\varepsilon,\mu)$, $m = \card{\St}$, and $\varepsilon$ is the false positive rate.
\end{theorem}
\begin{proof}
Each iteration of the loop in \Cref{alg:shmap} has a collision test which is Bernoulli distributed with a probability of success $p(m,\varepsilon,\mu)$. We are interested in the random length $\rv{N}$ of the bit string when this outcome occurs.

For the random variable $\rv{N}$ to realize a value $n$, every bit string smaller than length $n$ must fail and a bit string of length $n$ must succeed. There are $2^n-1$ bit strings smaller than length $n$ and each one fails with probability $q$, and so by the product rule the probability that they all fail is given by
\begin{equation}
\label{eq:N_pmf_1}
    q^{2^n - 1}\,.
\end{equation}

Given that every bit string smaller than length $n$ fails, what is the probability that every bit string of length $n$ fails? There are $2^n$ bit strings of length $n$, each of which fails with probability $q$ as before and thus by the product rule the probability that they all fail is $q^{2^n}$, whose complement, the probability that not all bit strings of length $n$ fail, is given by
\begin{equation}
\label{eq:N_pmf_2}
    1 - q^{2^n}\,.
\end{equation}
By the product rule, the probability that every bit string smaller than length $n$ fails and a bit string of length $n$ succeeds is given by the product of (\ref{eq:N_pmf_1}) and (\ref{eq:N_pmf_2}),
\begin{equation}
\label{eq:prob_is_mass}
    q^{2^n-1}\left(1 - q^{2^n}\right)\,.
\end{equation}
For \Cref{eq:prob_is_mass} to be a probability mass function, two conditions must be met. First, its range must be a subset of $[0,1]$. Second, the summation over its domain must be $1$.

The first case is trivially shown by the observation that $q$ is a positive number between $0$ and $1$ and therefore any non-negative power of $q$ is positive number between $0$ and $1$.

The second case is shown by calculating the infinite series
\begin{align}
    S &= \sum_{n=0}^{\infty} q^{2^n-1}\left(1-q^{2^n}\right)\\
      &= \sum_{n=0}^{\infty} q^{2^n-1} - q^{2^{n+1}-1}\,.
\end{align}
Explicitly evaluating this series for the first $4$ terms reveals a telescoping sum given by
\begin{equation}
    S = (1 - q) + (q - q^3) + (q^3 - q^7) + (q^7 - q^{15}) + \cdots,
\end{equation}
where everything cancels except $1$.
\end{proof}

The expected encoding size is given by the following theorem.
\begin{theorem}
\label{thm:open_expect}
The expected \emph{in-place} (no decoding step) encoding size of the \gls{gls-shs} $\Sp$ is given by
\begin{equation}
    \mathcal{O}(1) + \sum_{n=1}^{\infty} q^{2^n-1},
\end{equation}
where $q=1 - p(m,\varepsilon,\mu)$, $m = \card{\St}$, $\varepsilon$ is the false positive rate, and $\mu$ is the average bit length of the values in the key-value pairs.
\end{theorem}
\begin{proof}
The hash function $\hash_{\St}$ given by \Cref{alg:ph} finds a bit string of random length $\rv{N} = n$. Thus the encoding size is a discrete random variable given by
\begin{equation}
    \mathcal{O}(1) + \rv{N},
\end{equation}
where the constant is the space required to encode the random oracle. The expected encoding size is the expectation given by
\begin{align}
    \mathcal{O}(1) + \expectation[\rv{N}]\,.
\end{align}
By \Cref{thm:N_pmf}, the random variable $\rv{N}$ has a probability mass function given by
\begin{equation*}
    \rv{N} \sim \pmf_{\rv{N}}(n \given m,r) = q^{2^n-1}\left(1 - q^{2^n}\right),
    \tag{\ref{eq:N_pmf} revisited}
\end{equation*}
and thus the expectation of $\rv{N}$ is given by
\begin{align}
    \expectation\!\left[\rv{N}\right]
    &= \sum_{n=0}^{\infty} n q^{2^n-1}\left(1 - q^{2^n}\right)\\
    &= \sum_{n=0}^{\infty} n \left(q^{2^n-1} - q^{2^{n+1}-1}\right)\,.
\end{align}
Explicitly evaluating this series for the first $4$ terms reveals a converging sum given by
\begin{align}
    \expectation\!\left[\rv{N}\right]
        &= 0 + (q - q^3) + 2(q^3 - q^7) + 3(q^7 - q^{15}) + \cdots\\
        &= q + q^3 + q^7 + q^{15} + \cdots\,.
\end{align}
\end{proof}


\printglossary
\bibliography{references}
\end{document}
