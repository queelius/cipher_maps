\documentclass{article}
\usepackage[english]{babel}
\usepackage{graphicx}
\graphicspath{img}
\usepackage[activate={true,nocompatibility},final,tracking=true,kerning=true,spacing=true,factor=1100,stretch=10,shrink=10]{microtype}
\microtypecontext{spacing=nonfrench}
\usepackage[]{algorithm2e}
\usepackage{caption}
\usepackage{bbm}
\usepackage{mathtools}
\usepackage{commath}
\usepackage{amsmath}
\usepackage{subfiles}
\usepackage[xindy,toc,acronyms,symbols]{glossaries}
\usepackage{latexsym}
\usepackage{amsthm}
\usepackage{amssymb}
\usepackage{pgfplots}
\usepackage{wasysym}
\usepackage{hyperref}
\usepackage{tikz}
\usepackage{tikzscale}
\usepackage[square,numbers]{natbib}
\usepackage[utf8]{inputenc}
\usepackage[T1]{fontenc}
\usepackage{cleveref}
\usepackage{siunitx}
\usepackage[section]{placeins}

\hypersetup{
    pdftitle={The oblivious map abstract data type},             % title
    pdfauthor={Alexander Towell},               % author
    pdfsubject={computer science},              % subject of the document
    pdfkeywords={
        probabilistic data structure,
        approximate set,
        approximate domain},                    % keywords
    colorlinks=false,                           % false: boxed links;
    citecolor=green,                            % color of links to 
    filecolor=magenta,                          % color of file links
    urlcolor=green                              % color of external
}



\title
{
    The \emph{oblivious map} abstract data type
}
\author
{
    Alexander Towell\\
    \texttt{atowell@siue.edu}
}
\date{}

\begin{document}
\maketitle
\begin{abstract}
We define the semantics of the \emph{oblivious map}, which is a \emph{probabilistic} map that \emph{approximates} another map with a false positive rate $\varepsilon$ and false negative rate $\eta$ with the \emph{positive} ($\eta = 0$) and negative ($\varepsilon = 0$) oblivious maps being special complementary cases. As approximate maps, all the results that apply to approximate maps apply to them. We derive a theoretical data structure that implements the abstract data type of the \emph{oblivious map}, denoted the \gls{gls-shmap}. We explore its theoretical properties and conjecture that it obtains the lower-bound on space complexity when the inputs and outputs are of arbitrary length with \emph{false positive} and \emph{false negative} domain error rates given respectively by $\varepsilon$ and $\eta$. Additionally, we demonstrate an application, rank-ordered \emph{Encrypted Search}.
\end{abstract}

\tableofcontents
\listoffigures
\listofalgorithms

\subfile{sections/intro}
\subfile{sections/oblivious_map}
\subfile{sections/singular_hash_map}
\subfile{sections/apps}
\subfile{sections/appendix}

\printglossary
\bibliography{references}
\end{document}
