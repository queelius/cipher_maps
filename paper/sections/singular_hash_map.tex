\documentclass[ ../main.tex]{subfiles}
\providecommand{\mainx}{..}
\begin{document}
\section{The \emph{Singular Hash Map}}
The \emph{Singular Hash Set} is a theoretical data structure that provides an \emph{optimal} implementation of the \emph{oblivious map} abstract data type. A more practical implementation is given by the \emph{Perfect Map Filter}\cite{pmf}.

We consider sets in which the universe of elements is given by the \emph{countably infinite} set of all bit strings.
\begin{definition}
The countably infinite set of all bit strings is denoted by
\begin{equation}
    \cisb = \left\{b \colon b \in \left\{0,1\right\}^*\right\}\,,
\end{equation}
where $*$ denotes all non-negative integers.
\end{definition}
A \emph{finite} subset of $\cisb$ is given by the following definition.
\begin{definition}
The \emph{finite set} of all bit strings of length $n$ is denoted by
\begin{equation}
    \cisb_n = \left\{b \colon b \in \left\{0,1\right\}^n\right\}
\end{equation}
with a cardinality given by
\begin{equation}
    \card{\cisb_n} = 2^n\,.
\end{equation}
\end{definition}

A \emph{hash function} is related to countable sets $\cisb$ and $\cisb_n$ and is given by the following definition.
\begin{definition}
A \emph{hash function} $\hash \colon \cisb \mapsto \cisb_n$ is a function such that all bit strings of arbitrary-length are mapped (hashed) to bit strings of fixed-length $n$.
\end{definition}
For a given $x \in \cisb$, $y = \hash{x} \in \cisb_n$ is denoted the \emph{hash} of $x$.









The \gls{gls-shmap} depends on a random oracle as given by the following definition.
\begin{definition}
\label{def:randomoracle}
A random oracle, denoted by $\ro \colon \cisb \mapsto \cisb_n$, is a theoretical \emph{hash function} whose output is uniformly distributed over the elements of $\cisb_n$.
\end{definition}
\begin{assumption}
The hash function $\hash$ approximates a random oracle $\ro$ and has a space complexity given by $\mathcal{O}(1)$.
\end{assumption}

The bit length of objects is given by the following definition.
\begin{definition}
The bit length of an object $x$ is denoted by
\begin{equation}
    \BL(x)\,.
\end{equation}
\end{definition}
For example, the bit length of an object $x \in \cisb_n$ is given by $\BL(x) = n$.

\begin{algorithm}[h]
    \caption{Theoretical implementation of \protect\MakeApproxMap}
    \label{alg:shmap}
    \SetKwProg{func}{function}{}{}
    \KwIn
    {
        $\Fun = \{\left(x,\Fun(x)\right) \colon x \in \X \}$ is the function defined as a set of ordered pairs, $\varepsilon$ is the \emph{false positive} domain error rate, and $\eta$ is the \emph{false negative} domain error rate.
    }
    \KwOut
    {
        An \emph{oblivious} function $\Afun$ with an approximate domain with a \emph{false positive} rate $\varepsilon$ and \emph{false negative} rate less than or equal to $\eta$.
    }
    \SetKwData{found}{found}
    \SetKwData{truecode}{true\_code}
    \SetKwData{code}{test\_code}
    \func{\MakeApproxMap{$\Fun$, $\varepsilon$}}
    {
        $\Mt_{\cisb \times \cisb} \gets \left\{\left(\Encode_{\X \mapsto \cisb}\!\left(x\right),\Encode_{\Y \mapsto \cisb}(\Fun(x))\right) \colon x \in \X\right\}$\;
        $p \gets (1 - \eta) \card{\X}$\;
        $k \gets -\log_2 \varepsilon$\;
        \For{$n \gets 0$ \KwTo $\infty$}
        {
            $\found \gets \True$\;
            \For{$j \gets 1$ \KwTo $2^n$}
            {
                $b_n \gets$ draw random bit string of length $n$ from $\cisb_n$ without replacement\;
                $\hash(z) \equiv \ro(z \catop b_n)$\;
                \tcp{A minimum of $p$ inputs must have the same
                hash.}
                $\mathbb{M} \gets \text{all subsets of $p$ elements from $\Mt_{\cisb \times \cisb}$}$\;
                \For{$\mathbb{P} \in \mathbb{M}$}
                {
                    \tcp{Let the $p$ elements in $\mathbb{P}$ be denoted by $\left(x^{(1)}, y^{(1)}\right),\ldots,\left(x^{(p)},y^{(p)}\right)$.}
                    $h_k \gets \hash(x) \mod k$\;
                    \For{$i \gets 2$ \KwTo $p$}
                    {
                        \If{$\hash(x^{(i)}) \mod k \neq h_k$}
                        {
                            $\found \gets \False$\;
                        }
                        \tcp{$\Encode_{\mathbb{Y} \mapsto \cisb}$ is a {self-delimiting} coder so we only check that the first $\BL(y^{(i)})$ bits of the hash are equal to $y^{(i)}$.}
                        $\code \gets \hash(1 \catop x_i) \mod \BL(y^{(i)})$\;
                        \If{$\code \neq y^{(i)}$}
                        {
                            \label{line:selfterm}
                            $\found \gets \False$\;
                        }
                    }
                    \If{\found}
                    {
                        \tcp{This tuple codes the \gls*{gls-shmap}.}
                        \Return $(h_k,b_n)$\;
                    }
                }
            }
        }
    }
\end{algorithm}

\begin{algorithm}[h]
    \caption{Theoretical implementation of \protect\Find}
    \label{alg:shmapfind}
    \SetKwProg{func}{function}{}{}
    \KwIn
    {
        $\Mp \in \mathbb{X} \times \mathbb{Y}$ is the \emph{\gls{gls-shmap}} with a \emph{\gls{gls-fprate}} $\varepsilon$.
    }
    \KwOut
    {
        If $x \in \mathbb{X}$ is a key in $\Mt$, then the \emph{value} in $\mathbb{Y}$ mapped to by $x$ is returned. Otherwise, with probability $\varepsilon$, $k$ is a false positive and maps to a \emph{random} bit string and with probability $1-\varepsilon$, the \emph{null value} is returned.
    }
    
    \SetKwData{Code}{code}
    \SetKwData{Success}{success}
    \func{\Find{$\Mp$, $x$}}
    {
        \tcp{The \emph{\gls{gls-shmap}} is coded by the tuple $(h_1,b_n)$ as given by \Cref{alg:shmap}.}
        $\ell \gets \BL(h_1)$\;
        $\Code_x \gets \Encode_{\mathbb{X} \mapsto \cisb}(x)$\;
        \If{$\hash(\Code_x) \mod \ell \neq h_1$}
        {
            \Return \nullvalue\;
        }
        
        
        $\Code_{x'} \gets \hash(1 \catop x) \mod j$\;
        \tcp{We assume the true code associated with $x$ is the result of a \emph{self-delimiting} coder which can be decoded by $\Decode \colon \cisb \mapsto \mathbb{Y}$.}
        $(\Success, y) \gets \Decode_{\cisb \mapsto \mathbb{Y}}(\Code_y)$\;
        \If{\Success}
        {
            \tcp{Since $k$ may be a false positive, $v$ may be a random bit string with a length $\ell \in [c,N], 0 \leq c \leq N$, where $c$ depends upon the self-delimiting coder.}
            \Return $y$\;
        }
    }
\end{algorithm}


\begin{figure}
    \centering
    %%% Creator: Inkscape inkscape 0.92.1, www.inkscape.org
%% PDF/EPS/PS + LaTeX output extension by Johan Engelen, 2010
%% Accompanies image file 'shs.pdf' (pdf, eps, ps)
%%
%% To include the image in your LaTeX document, write
%%   \input{<filename>.tex}
%%  instead of
%%   \includegraphics{<filename>.pdf}
%% To scale the image, write
%%   \def\svgwidth{<desired width>}
%%   \input{<filename>.pdf_tex}
%%  instead of
%%   \includegraphics[width=<desired width>]{<filename>.pdf}
%%
%% Images with a different path to the parent latex file can
%% be accessed with the `import' package (which may need to be
%% installed) using
%%   \usepackage{import}
%% in the preamble, and then including the image with
%%   \import{<path to file>}{<filename>.pdf_tex}
%% Alternatively, one can specify
%%   \graphicspath{{<path to file>/}}
%% 
%% For more information, please see info/svg-inkscape on CTAN:
%%   http://tug.ctan.org/tex-archive/info/svg-inkscape
%%
\begingroup%
  \makeatletter%
  \providecommand\color[2][]{%
    \errmessage{(Inkscape) Color is used for the text in Inkscape, but the package 'color.sty' is not loaded}%
    \renewcommand\color[2][]{}%
  }%
  \providecommand\transparent[1]{%
    \errmessage{(Inkscape) Transparency is used (non-zero) for the text in Inkscape, but the package 'transparent.sty' is not loaded}%
    \renewcommand\transparent[1]{}%
  }%
  \providecommand\hashtatebox[2]{#2}%
  \ifx\svgwidth\undefined%
    \setlength{\unitlength}{214.63718569bp}%
    \ifx\svgscale\undefined%
      \relax%
    \else%
      \setlength{\unitlength}{\unitlength * \real{\svgscale}}%
    \fi%
  \else%
    \setlength{\unitlength}{\svgwidth}%
  \fi%
  \global\let\svgwidth\undefined%
  \global\let\svgscale\undefined%
  \makeatother%
  \begin{picture}(1,0.55063496)%
    \put(0.38,0.078){\color[rgb]{0,0,0}\makebox(0,0)[lb]{\smash{$h(x_2 + b) \mod 4$}}}%
    \put(0.08843285,0.51926543){\color[rgb]{0,0,0}\makebox(0,0)[lb]{\smash{set $\mathbb{S}$}}}%
    \put(0,0){\includegraphics[width=\unitlength,page=1]{fig_shs.pdf}}%
    \put(0.09242613,0.24054801){\color[rgb]{0,0,0}\makebox(0,0)[lb]{\smash{$x_1$}}}%
    \put(0.12846631,0.36240699){\color[rgb]{0,0,0}\makebox(0,0)[lb]{\smash{$x_3$}}}%
    \put(0.13833525,0.11371431){\color[rgb]{0,0,0}\makebox(0,0)[lb]{\smash{$x_2$}}}%
    \put(0.92073981,0.50635972){\color[rgb]{0,0,0}\makebox(0,0)[lt]{\begin{minipage}{0.53112884\unitlength}\raggedright \end{minipage}}}%
    \put(0,0){\includegraphics[width=\unitlength,page=2]{fig_shs.pdf}}%
    \put(0.41093171,0.41327568){\color[rgb]{0,0,0}\makebox(0,0)[lb]{\smash{$h(x_3 + b)$}}}%
    \put(0.35,0.275){\color[rgb]{0,0,0}\makebox(0,0)[lb]{\smash{$h(x_1 + b)}$}}}%
    \put(0,0){\includegraphics[width=\unitlength,page=3]{fig_shs.pdf}}%
    \put(0.81940596,0.29320931){\color[rgb]{0,0,0}\makebox(0,0)[lb]{\smash{$0\,1\,0\,0$}}}%
    \put(0.76349761,0.2128412){\color[rgb]{0,0,0}\makebox(0,0)[lb]{\smash{$\fprate = 2^{-4}$}}}%
  \end{picture}%
\endgroup%
    \caption{\emph{Singular Hash Map}}
    \label{fig:shmap}
\end{figure}

We consider a family of \emph{approximate maps} which are given by the output of the random oracle applied to the input $x \in \Mt$ concatenated with a bit string $b \in \cisb$ such that all $x \in \St$ map to the same hash. We describe the generative algorithm in \Cref{alg:ph}. Note that in \Cref{alg:shmap}, the concatenation of two bit strings $x$ and $y$ is denoted by $x \catop y$.


In \Cref{alg:shmap} on \Cref{line:selfterm}, the value $v_i$ has a \emph{self-terminating} encoding. Thus, we need only check that the bit string that encodes $v_i$ is equal to a hash of the key (concatenated with another $1$ and another bit string $b$).


The space required for the \gls{gls-shmap} found by \Cref{alg:shmap} is of the order of the length $n$ of the bit string $b$. Therefore, for space efficiency, the algorithm exhaustively searches for a bit string in the order of increasing size $n$.

\subsection{Theoretical analysis}
The \gls{gls-set} proves that the keys in $\Mp$ are an \emph{approximate set} of the keys in $\Mt$.

\begin{theorem}
The probability that a particular bit string $b \in \cisb$ successfully codes the \gls{gls-shmap} is given by
\begin{equation}
    p(m,\varepsilon,\mu) = \frac{\varepsilon^{m-1}}{2^{m \mu}}\,.
\end{equation}
\end{theorem}
\begin{proof}
In \Cref{alg:shmap}, for a particular bit string $b_n$, the joint probability that every key in $\Mt$ collides and for each key $k$ in $\Mt$, the concatenation of $1$ and $k$ has a hash in which the first $\BL(v)$ bits collide with mapped value $v$ is given by the following reasoning.

Suppose we have a set $\Mt = \{(k_1,v_1),\ldots,(k_m,v_m)\}$ and $k_1$ hashes to $y = \hash(k_1) \mod t$, where $\hash \colon \cisb^* \mapsto \cisb_N$ is a random hash function that uniformly distributes over its domain of $2^N$ possibilities and $N < \max{\BL(v_1),\ldots,\BL(v_m)}$.

The probability that $\hash(1 \catop k_1) \mod \BL(v_1) = v_1$ is given by
\begin{equation}
    \frac{1}{2^{\BL(v_1)}}\,.
\end{equation}
Since $y$ is a particular element in $\cisb_k$, the probability that $k_j, j \neq 1$, hashes to $y$ is given by
\begin{equation}
    \frac{1}{2^k}\,.
\end{equation}
The probability that $1 \catop k_j, j \neq 1$ hashes to $v_j$ is given by
\begin{equation}
    \frac{1}{2^{\BL(v_j)}}\,.
\end{equation}
Since $\hash$ is a random hash function, every one of these probabilities are independent, and thus the joint probability is just the product given by
\begin{align}
    p &= \frac{1}{2^{k(m-1)}} \prod_{j=1}^{m} \frac{1}{2^{\BL(v_j)}}\\
      &= \varepsilon^{m-1} \frac{1}{2^{\sum_{j=1}^{m} \BL(v_j)}}\,.
\end{align}
The average bit length per key is given by
\begin{equation}
    \mu = \frac{1}{m} \sum_{j=1}^{m} \BL(v_j)
\end{equation}
and thus the probability $p$ may be parameterized as
\begin{equation}
    p = \frac{\varepsilon^{m-1}}{2^{m \mu}}\,.
\end{equation}
\end{proof}

\begin{theorem}
Given \emph{false positive} and \emph{false negative} domain error rates given by $\varepsilon$ and $\eta$ respectively, and a mean bit length $\mu$ of encodings of elements in the image of $\Afun$, the \emph{expected} bit length of the Singular Hash Map asymptotically obtains a space complexity with a lower-bound given by
\begin{equation}
    -(1 - \eta) \log_2 \varepsilon + (1 - \eta) \mu \; \si{bits \per element}\,,
\end{equation}
which occurs when the encodings have uniformly distributed bit lengths, and an upper-bound given by
\begin{equation}
    -(1 - \eta) \log_2 \varepsilon + \mu \; \si{bits \per element}\,,
\end{equation}
which occurs when the encodings have degenerate bit lengths. The greater the entropy of the bit lengths, the smaller the expected bit length given a mean of $\mu$. 
\end{theorem}
\begin{proof}
The space required for the \gls{gls-shs} found by \Cref{alg:makeset} is of the order of the length $n$ of the bit string $b_n$. Therefore, for space efficiency, the algorithm exhaustively searches for a bit string in the order of increasing size $n$.

We are interested in the first case when a success occurs, which is a geometric distribution with probability of success $p$ as given by
\begin{equation}
    \rv{Q} \sim \geodist\!\left(p = \frac{\varepsilon^{m-1}}{2^{m \mu}}\right)\,.
\end{equation}
The expected number of trials for the geometric distribution is given by
\begin{equation}
\label{eq:exp_trials}
    \expectation\left[\rv{Q}\right] = \frac{2^{m \mu}}{\varepsilon^{m-1}}\,.
\end{equation}
By \Cref{def:mapping}, the \nth trial uniquely maps to a bit string of length $m = \lfloor \log_2 n \rfloor$. Thus, the expected bit length is given approximately by
\begin{equation}
    \expectation[\log_2 \rv{Q}] = \log_2 \frac{2^{m \mu}}{\varepsilon^{m-1}}\; \si{bits}\,.
\end{equation}
Dividing by $m$ and simplifying results in
\begin{equation}
    -\frac{m-1}{m} \log_2 \varepsilon + \mu\; \si{bits \per element}\,.
\end{equation}
Asymptotically, as $m \to \infty$, the bits per element goes to
\begin{equation}
    -\log_2 \varepsilon + \mu\,.
\end{equation}
\end{proof}

The smaller the bit lengths generated by the encoder, the smaller the \emph{Singular Hash Map}. However, optimal encoders \emph{reveal} information about the values in the \emph{Singular Hash Map}, which goes against the principle of the \emph{obvlivious function}. For instance, if each input $x$ mapped to a list of numbers, then an optimal encoder will generate codes that minimize the length of the list, say a Huffman code. However, by analyzing the Huffman codes produced by the encoder, we may infer the distribution of the lists. Thus, for the sake of maintaining an \emph{oblivious} state, the encoder should generate codes as though the values were \emph{uniformly} distributed.

\subsection{Entropy}


\end{document}