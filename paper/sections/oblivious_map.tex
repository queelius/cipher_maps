\documentclass[ ../main.tex]{subfiles}
\providecommand{\mainx}{..}
\begin{document}
\section{Oblivious maps}
\label{sec:map}
A set is an unordered collection of distinct elements, typically from some implicitly understood universe.
A countable set is a \emph{finite set} or a \emph{countably infinite set}. A \emph{finite set} has a finite number of elements, such as $\{ 1, 3, 5 \}$, and 
a \emph{countably infinite set} can be put in one-to-one correspondence with the set of natural numbers, $\{1,2,3,4,5,\ldots\}$.
The cardinality of a set $\mathbb{A}$, denoted by $|\mathbb{A}|$, is a measure on the number of elements in the set, e.g., $|\{a,b\}|=2$.

A map represents a \emph{many}-to-\emph{one} relationship. A map that associates elements in $\mathbb{X}$ to elements in $\mathbb{Y}$ has a type denoted by $\mathbb{X}\mapsto\mathbb{Y}$.
For a map of type $\mathbb{X} \mapsto \mathbb{Y}$, we denote $\mathbb{X}$ the \emph{input} and $\mathbb{Y}$ the output.
Typically, it is relatively easy to find which output is associated with a given input, but the inverse operation, determining which inputs are associated with a given output is computationally harder.
Of course, this is not necessarily the case, and mathematically the map is just a one-to-many relation over $\mathbb{X} \times \mathbb{Y}$.

For instance, \Cref{tbl:tabfunc} depicts a function over a finite domain of $n$ elements, where each input $x \in X$ is associated with a single output $y \in Y$, i.e., $y = \Fun(x)$.

\begin{table}
    \centering
    \caption{Function $\operatorname{negate} \colon \{0,1\} \mapsto \{0,1\}$}
    \label{tbl:tabfunc}
    \begin{tabular}{|c c|} 
        \hline
        Input ($\{0,1\}$) & Output ($\{0,1\}$)\\
        \hline
        $0$ & $1$\\
        $1$ & $0$\\    
        \hline
    \end{tabular}
\end{table}

The input does not need to be a simple set like natural numbers, but rather can be any type of set, such as a set of pairs as given in \Cref{tbl:tabfun2}.

\begin{table}
\centering
\caption{Function $\operatorname{f} \colon \X \mapsto \Y$\\defined over a \emph{finite} domain $\X$}
\label{tbl:tabfunc2}
\begin{tabular}{|c c|} 
\hline
Input ($\{0,1\} \times \{0,1\}$) & Output ($\{0,1\}$)\\
\hline
    $(0,0)$ & $0$\\
    $(0,1)$ & $1$\\
    $(1,0)$ & $1$\\
    $(1,1)$ & $0$\\    
\hline
\end{tabular}
\end{table}


Since we are interested in constructing maps in computer memory, we must have some way to represent them.
One technique may be given by the following table.

\begin{table}
\centering
\caption{Function $\operatorname{f} \colon \{0,1\} \mapsto \{0,1\}$ where $b : \{0,1\} \mapsto \{0,1\}^*$ is the encoder for $\mathbb{Y}$}
\label{tbl:tabfunc}
\begin{tabular}{|c c c|} 
\hline
Input ($\mathbb{X}$) & Output ($\mathbb{Y}$) & Encoding ($\cisb$)\\
\hline
    $x_1$ & $y_1$ & $b_1$\\
    $x_2$ & $y_2$ & $b_2$\\
    $x_3$ & $y_3$ & $b_3$\\
    $\vdots$ & $\vdots$ & $\vdots$\\
    $x_n$ & $y_n$ & $b_n$\\
\hline
\end{tabular}
\end{table}



The oblivious map is given by the following definition.
\begin{definition}
The \gls{gls-obliviousmap} abstract data type is a specialization of the approximate map. We denote an oblivious map of $\Mt$ by $\Mpp$. An \emph{oblivious map} $\Mpp$ of $\Mt$ must satisfy the following conditions:
\begin{enumerate}
    \item $\AT{f} \colon \AT{X \mapsto Y}$ is a random approximate map of $f \colon X \mapsto Y$ with a specified false positive rate $\varepsilon$.
    \item There is no way to efficiently iterate over the domain of definition $X$.
    \item If an element of $x \in X$ is not in the domain of definition, $f(x)$ is a random oracle over $Y$
    \item A particular mapping $y = f(x)$ may only be learned by evaluating the function $f(x)$.
    \item Any estimator of the cardinality of the domain of definition that only uses the information in an object's binary representation may only be able determine an approximate specifiable upper-bound $u \geq m$ as a function of the entropy and the \emph{efficiency} of the representation, i.e., the most efficient representation has a very precise estimator of the cardinality of the domain of definition.
\end{enumerate}
\end{definition}
In an \emph{oblivious map}, a mapping (row in the table) is only learned upon request.

By \cref{dummyref}, the space complexity of the \emph{oblivious map} with a false positive rate $\varepsilon > 0$ and a false negative rate $\eta = 0$ is given by the following postulate.
\begin{postulate}
Space complexity
\begin{equation}
    -\log_2 \varepsilon + \mu \; \si{bits \per value}\,,
\end{equation}
where $\varepsilon$ is the false positive rate and $\mu$ is the mean \emph{marginal} entropy that a positive key maps to.
\end{postulate}

The cardinality of $\Mt$ is uniformly distributed between $0$ and The upper-bound $u$. Thus, the entropy of the cardinality is $\log_2(u+1)$. Thus, entropy may be traded for efficiency. For the optimal oblivious map, the relative efficiency to an approximate map where the entropy of the values in the key-value pairs is on average $\mu$ per value is given by
\begin{equation}
    \frac{\log_2 \varepsilon + \mu}{\frac{u}{m}\log_2 \varepsilon + \mu}\,.
\end{equation}
If $u = m$, then its relative efficiency is $1$. If $\mu = 0$, e.g., an approximate set, then the relative efficiency is given by
\begin{equation}
    \frac{m}{u}\,.
\end{equation}

\subsection{Abstract data type}
A \emph{type} is a set and the elements of the set are called the \emph{values} of the type. An \emph{abstract data type} is a type and a set of operations on values of the type. For example, the \emph{integer} abstract data type is the set of all integers and a set of standard operations such as addition and subtraction.

A \emph{data structure} is a particular way of organizing data and may implement one or more abstract data types. An \emph{immutable} data structure has static state; once constructed, its state does not change until it is destroyed. The abstract data type of the \emph{immutable} approximate map is given by the following definition..
\begin{definition}
The \emph{static map} is an abstract data type of the \emph{immutable} map. Any of the following operations may be supported:
\begin{enumerate}
    \item \HasKey{$\Mp$,$k$}. Returns $\True$ if $(k,v') \in \Mt$ for any $v'$. Otherwise, \False.
    \item \Find{$\Mp$,$k$}. Returns $v$ if $(k,v) \in \Mt$ for any $v$. Otherwise, returns some unspecified $v'$ with probability $\varepsilon$ and $\nullvalue$ with probability $1 - \varepsilon$.
    \item \Cardinality{$\Mp$}. Returns the \emph{expected} cardinality $(1 - \varepsilon) \card{\Mt} + \varepsilon \card{\mathbb{X}}$.
    \item \Count{$\Mt$}. Returns $\card{\Mt}$.
    \item \fprate{$\Mp$}. Returns the \emph{\gls{gls-fprate}} of $\Mp$.
\end{enumerate}
\end{definition}
\end{document}